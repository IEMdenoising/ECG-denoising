\title{Brief Article} \documentclass[11pt, oneside]{article}   
\usepackage{geometry}               		
\geometry{letterpaper}                   		
%\usepackage[parfill]{parskip}    		% Activate to begin paragraphs with an empty line rather than an indent
%\usepackage{graphicx}				% Use pdf, png, jpg, or eps§ with pdflatex; use eps in DVI mode

\usepackage[T1]{fontenc}
\usepackage[cp1250]{inputenc}
%\usepackage{polski}

\usepackage[english,polish]{babel} % U\.zyj polskiego \l{}amania wyraz�w (zamiast domy\'clnego angielskiego).
\usepackage{polski} %\usepackage[utf8]{inputenc}
\usepackage{indentfirst}

\usepackage{mathtools}
\usepackage{amsfonts}
\usepackage{amsmath}
\usepackage{amsthm}
\usepackage{algorithm}
\usepackage[noend]{algpseudocode}

\algnewcommand\algorithmicOutput{\textbf{Wyj\'scie:}}
\algnewcommand\Output{\item[\algorithmicOutput]}

\algnewcommand\algorithmicInput{\textbf{Wej\'scie:}}
\algnewcommand\Input{\item[\algorithmicInput]}

\algnewcommand\algorithmicVar{\textbf{Zmienne pomocnicze:}}
\algnewcommand\Temporary{\item[\algorithmicVar]}



\title{Algorytm Nonlocal Means}
\author{M. P\k{e}kala, A.Potempa}
%\date{}							% Activate to display a given date or no date

\begin{document}
\maketitle
\section{Wst\k{e}p}

Algorytm Nonlocal Means najcz\k{e}\'sciej stosowany jest w przetwarzaniu obraz�w. W odr�\.znieniu od filtr�w wykorzystuj\k{a}cych algorytm local means, gdzie przetworzona warto\'s\'c piksela liczona jest jako \'srednia pikseli j\k{a} otaczaj\k{a}cych to w algorytmie nonloclal means warto\'s\'c piksela wyznaczana jest jako \'srednia ze wszystkich punkt�w obrazu z uwzgl\k{e}dnieniem wagi podobie\'nstwa pikseli w zale\.zno\'sci od przetwarzanego punktu. 

Ten algorytm mo\.ze by\'c te\.z stosowany do przetwarzania sygna\l{}�w 1D, co pozwala na zastosowanie go do filtracji sygna\l{}u EKG.
\section{Opis matematyczny}
\subsection{Algorytm Nonlocal Means}
Dla sygna\l{}u $v$, kt�ry jest oryginalnym, zaszumionym sygna\l{}em, warto\'s\'c przetworzonego sygna\l{}u $u$ wyra\.za si\k{e} wzorem:

\begin{equation}
\begin{split}\'n
\frac{1}{Z(s)}\displaystyle\sum_{t\in{N(s)}}w(s,t)v(t),
\end{split}
\end{equation}
gdzie $w(s,T)$ s\k{a} wagami, kt�re mierz\k{a} podobie\'nstwo pomi\k{e}dzy dwoma kwadratowymi wycentrowanymi obszarami wok�\l{} $s$ oraz $t$. Wagi mo\.zna zdefiniowa\'c przy pomocy r�wnania nr 2.  $Z(s)$ jest znormalizowan\k{a} sta\l{}\k{a}, gdzie dla ka\.zdego $s$ zachodzi zale\.zno\'s\'c $Z(s) = \sum_{t\in{N(s)}}w(s,t)$. $N(s)$ odpowiada otoczeniu punktu $s$, nazywanym cz\k{e}sto obszarem okna poszukiwa\'n. 

\begin{equation}
\begin{split}
w(s,t) = g_{h}(\displaystyle\sum_{\delta\in{\Delta}}G_{\sigma}(\delta)(v(s+\delta) - v(t+\delta))^2),
\end{split}
\end{equation}
$G_{\sigma}$ jest Gaussowskim j\k{a}drem przekszta\l{}cenia wariancji $\sigma^2$, $g_{h}$ jest ci\k{a}g\l{}\k{a} nierosn\k{a}c\k{a} funkcj\k{a} dla kt�rej $g_{h}(0) = 1$ a $lim_{x\to+\infty}g_{h}(x) = 0$, a $\Delta$ reprezentuje obszar zawieraj\k{a}cy otoczenie punktu $\delta$.

Podsumowuj\k{a}c algorytm Nonlocal Means odtwarza sygna\l{} poprzez przeprowadzenie wa\.zonej \'sredniej warto\'sci punkt�w uwzgl\k{e}dniaj\k{a}c przestrzenne podobie\'nstwo pomi\k{e}dzy punktami. Podobie\'nstwo jest liczone pomi\k{e}dzy r�wnymi obszarami, w kt�rych uchwycona jest lokalna struktura (geometria, tekstura). Nale\.zy zwr�ci\'c uwag\k{e}, \.ze punkty poza obszarem $N(s)$ nie maj\k{a} wp\l{}ywu na warto\'s\'c $u(s)$.
Zak\l{}adamy, \.ze okna poszukiwa\'n $N$ i obszary $\Delta$ maj\k{a} tak\k{a} sam\k{a} liczebno\'s\'c $(2K+1)^d$ i $(2P+1)^d$ dla $N = [-K,K]^d$ i $\Delta = [-P,P]^d$. 

W przedstawionym do tej pory opisie znalaz\l{}y si\k{e} parametry, za pomoc\k{a} kt�rych wykonano odszumianie przebiegu EKG. Jednym z nich jest bazowy parametr opisany jako po\l{}owa d\l{}ugo\'sci $P$, czyli po\l{}owa szeroko\'sci za\l{}amka $P$ o najwi\k{e}kszej amplitudzie.

Sygna\l{}owi poddawanemu analizie odpowiada parametr $N(s)$.

\subsection{Algorytm Fast Nonlocal Means}

 Algorytm NLM jest dosy\'c skomplikowany, w zwi\k{a}zku z czym podj\k{e}to wiele stara\'n, aby przyspieszy\'c jego dzia\l{}anie. Najcz\k{e}\'sciej stosowanym podej\'sciem jest Fast Noclocal Means Algorytm, zaproponowany przez Darbona. To podej\'scie znacz\k{a}co przyspiesza dzia\l{}anie algorytmu poprzez zmian\k{e} kolejno\'sci operacji tak aby wyeliminowa\'c zagnie\.zd\.zone p\k{e}tle. Dzi\k{e}ki temu sygna\l{} 1D o d\l{}ugo\'sci $N$ b\k{e}dzie mie\'c z\l{}o\.zono\'s\'c obliczeniow\k{a} r�wn\k{a} $O(2NM)$, podczas gdy przy tradycjonalnym podej\'sciu wynosi $O(L_{\Delta}NM)$.

Podstawowym elementem algorytmu FNL jest efektywne policzenie wag $w(s,t)$. Wyznaczenie wag jest najbardziej czasoch\l{}onnym krokiem przy generowaniu przetworzonego sygna\l{}u $u$. Przy podej\'sciu 1D zak\l{}adamy, \.ze $\Omega = [0,n-1]$ dla sygna\l{}u $n$ elementowego. 
Maj\k{a}c wektor translacji $d_{x}$ wprowadzamy nowy sygna\l{} $S_{d_{x}}$ jako:
\begin{equation}
\begin{split}
S_{d_{x}}(p) = \displaystyle\sum_{k=0}^{p}(v(k)-v(k+d_{x}))^2 , p\in{\Omega}
\end{split}
\end{equation}

$S_{d_{x}} $ odpowiada dyskretnemu ca\l{}kowaniu kwadratu r�\.znic pomi\k{e}dzy sygna\l{}em $v$ a jego przesuni\k{e}ciem o wektor $d_{x}$. W sygna\l{}ach 1D mamy obszary $\Delta = [-P,P]$. J\k{a}dro przekszta\l{}cenia Gaussowskiego zamienia si\k{e} na sta\l{}\k{a}, co nie ma zauwa\.zalnego wp\l{}ywu na sygna\l{}, dzi\k{e}ki czemu r�wnanie 2 mo\.zna zapisa\'c w postaci: $w(s,t) = g_{h}(\sum_{\delta\in{\Delta}}(v(s+\delta_{x}) - v(t+\delta_{x}))^2)$. Zak\l{}adaj\k{a}c, \.ze $d_{x} = (t-s)$, a $p=s+\delta_{x}$ to r�wnanie $w(s,t)$ mo\.zna przeparametryzowa\'c do postaci $w(s,t) = g_{h}(\sum_{p=s-P}^{s+P}(v(p) - v(p+d_{x}))^2)$. Je\.zeli rozdzielimy sum\k{e} i u\.zyjemy r�wnania nr 3 to uzyskamy: 

\begin{equation}
\begin{split}\l{}
w(s,t) = g_{h}(S_{d_{x}}(s+P)-S_{d_{x}}(s-P))
\end{split}
\end{equation}
R�wnanie nr 4 jest niezale\.zne od t z za\l{}o\.zeniem, \.ze wielko\'s\'c $S_{d_{x}}$ jest znana. Jest to kluczowe r�wnanie, kt�re pozwala na wyznaczenie wag dla pary punkt�w w sta\l{}ym czasie. 

Podsumowuj\k{a}c, dzia\l{}anie algorytmu mo\.zna opisa\'c jako  wyznaczenie warto\'sci $S_{d_{x}}$ z wykorzystaniem r�wnania 3, nast\k{e}pnie wyznaczane s\k{a} wagi korzystaj\k{a}c z r�wnania 2 i 4, a na koniec przeprowadzana jest filtracja na podstawie r�wnania 1. Procedura jest powtarzana dla wszystkich mo\.zliwych translacji wyznaczonych przez wymiar okna poszukiwa\'n $N$.
 
\section{Implementacja algorytmu}

\begin{algorithm}
\caption{Algorytm Fast-NLM }
\begin{algorithmic}[1]
\Input $v,K,P,h$
\Output $u$
\Temporary $S_{d_{x}}, Z, M$
\State Inicjalizacja $u$, $M$, $Z$ jako $0$.
\ForAll{$d_{x}\in[-K,K]$}
\State {Obliczenie $S_{d_{x}}$ na podstawie r�wnania 3}
\ForAll {$s\in[0,n-1] $}
\State {Obliczenie wag w na podstawie r�wnania 2 i 4}
\State {$u(s)\leftarrow u(s) + w\cdot u(s+d_{x})$}
\State{$M(s) = max(M(s),w)$}
\State {$Z(s)\leftarrow Z(s) + w$}
\EndFor
\EndFor
\ForAll {$s\in[0,n-1] $}
\State {$u(s)\leftarrow u(s) + M(s) \cdot v(s)$}
\State {Zastosowanie normalizacji sygna\l{}u wyj\'sciowego}
\State {$u(s)\leftarrow \frac{u(s)}{Z(s) + M(s)} $}
\EndFor
\State \textbf{return} $u$
\end{algorithmic}
\end{algorithm}

Do implementacji algorytmu zosta\l{}y u\.zyte 4 paramaetry wej\'sciowe, oznaczone kolejno jako $v$, $K$, $P$ oraz $h$. Symbolem $v$ oznaczono sygna\l{} wej\'sciowy, podawany na wej\'scie implementowanej funkcji. U\.zyty  parametr $h$, decydowa\l{} o stopniu wyg\l{}adzenia sygna\l{}u po filtracji. Zbyt ma\l{}a warto\'s\'c tego parametru oznacza zbyt du\.zy wp\l{}yw szumu, co skutkuje niedostateczn\l{}ym u\'srednianiem sygna\l{}u. Efektem przyj\k{e}cia za parametr $h$ zbyt du\.zej warto\'sci jest zmiana wygl\k{a}du przebieg�w sygna\l{}u EKG, kt�ra spowoduje \.ze wiele sygna\l{}�w b\k{e}dzie do siebie podobnych mimo wyst\k{e}puj\k{a}cych r�\.znic. Pierwszy z dw�ch kolejnych parametr�w- $K$, zosta\l{} wykorzystany do stworzenia wektora b\k{e}d\k{a}cego zbiorem wed\l{}ug kt�rego iterowane by\l{}y kolejne przej\'scia algorytmu. Parametr $P$ odpowiada\l{} natomiast za s\k{a}siedztwo w kt�rym wyszukiwane b\k{e}d\k{a} kolejne pr�bki. Odpowiedni dob�r zakresu s\k{a}siedztwa wp\l{}ywa m.in. na szybko\'s\'c przeszukiwania sygna\l{}u co ma bezpo\'sredni wp\l{}yw na wydajno\'s\'c algorytmu. Zbyt du\.zy zakres s\k{a}siedztwa sprawi, \.ze czas kompilacji zostanie wydlu\.zony, nawet w przypadku szybkiego algorytmu. Dob�r odpowiednio du\.zego $P$ umo\.zliwi por�wnanie ze sob\k{a} zespo\l{}�w QRS wyst\k{e}puj\k{a}cych w zakresie parametru $P$.

\end{document}  